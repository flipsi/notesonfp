%!TEX root = fp.tex

% Author: Philipp Moers <soziflip funny character gmail dot com>

\chapter{Haskell Ramp-Up} % (fold)
\label{cha:haskell_ramp_up}

(Read $\equiv$ as 'denotes the same value as')

\begin{itemize}
    \item Apply f to value e: \codeline{f e} (juxtaposition, 'apply', binary operator \textvisiblespace, Haskell speak: infixL 10 \textvisiblespace)
    \item \textvisiblespace\ has max precedence (10): \codeline{f e$_1$ + e$_2$} $\equiv$ \codeline{(f e$_1$) + e$_2$}
    \item \textvisiblespace\ associates to the left: \codeline{g f e} $\equiv$ \codeline{(g f) e} \\ ('\codeline{(g f)}' is a function) \item Function composition: \begin{itemize}
        \item \codeline{( g . f ) e} $\equiv$ \codeline{g (f e)} \\ (. is something like mathematical $\circ$ 'after')
        \item Alternative 'apply'-operator \codeline{\$} (lowest precedence, associates to the right, infixR $\emptyset\ \$$):\\
            \codeline{g \$ f \$ e} $\equiv$ \codeline{g \$ (f \$ e)} $\equiv$ \codeline{g (f e)}
        \item Prefix application of binary infix operator $\otimes$: \codeline{$(\otimes)$ e$_1$ e$_2$} $\equiv$ \codeline{e$_1$ $\otimes$ e$_2$} 
        \item Infix application of binary function f: \codeline{e$_1$ `f` e$_2$} $\equiv$ \codeline{f e$_1$ e$_2$}:
        \begin{itemize}
            \item \codeline{1 `elem` [1,2,3]}   ($1 \in \{1,2,3\}$)
            \item \codeline{n `mod` m}
            \item \dots
        \end{itemize}
        \item User defined operators, built from symbols \\ ! \# \$ \% \& * + / < = > ? \@ \textbackslash \string^ \textbar $\sim$:.
    \end{itemize}
\end{itemize} 


% chapter haskell_ramp_up (end)


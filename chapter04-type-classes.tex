%!TEX root = fp.tex

% Author: Philipp Moers <soziflip funny character gmail dot com>

\chapter{Type Classes} % (fold)
\label{cha:type_classes}

A \textbf{type class} C defines a family of type signatures („methods“) which all \textbf{instances} of C must implement.

\begin{codebox}[haskell]
class C a where
    f_1 :: t_1
    ...
    f_n :: t_n
\end{codebox}
The \codeline{t\_i} \underline{must} mention a.\\
For any \codeline{f\_i} the class may provide default implementations. \\
We have \codeline{f\_i :: C a => t\_i} (read „if a is instance C then f\_i has type t\_i“).\\
\codeline{C a} is called \textbf{class constraint}.

\textit{Example}:

\begin{codebox}[haskell]
class Eq a where
    (==) :: a -> a -> Bool
    (/=) :: a -> a -> Bool
    x == y = not (x /= y)
    x /= y = not (x == y)
\end{codebox}
(These are default implementations. To redefine one of them is sufficient.)

% \codefile{haskell}{caption={type-classes.hs}, label=type-classes}{../material/type-classes.hs}


\section{Class Inheritance}

\begin{itemize}
    \item Defining \codeline{class (c$_1$ a, c$_2$ a, ...) => C a where ...} makes type class C a subclass of the C$_i$.\\
    \item \codeline{C a => t} implies \codeline{C$_1$ a, C$_2$ a ...}.
\end{itemize}

\vspace{9pt}
% \includegraphics[scale=1.0]{type-class-inheritance.png}
\vspace{9pt}


\section{Class Instances}

If type t implements the methods of class C, t becomes an \textbf{instance of} C:
\begin{codebox}[haskell]
instance C t where
    f_1 = <def of f_1>
    ...
    f_n = <def of f_n>
\end{codebox}
(All defs of f\_i may be provided, minimal complete definition \underline{must} be provided.)
Class constraint C t is satisfied from now on.

\textit{Example}:

\begin{codebox}[haskell]
instance Eq Bool where
    x == y = x && y || (not x && not y)
\end{codebox}

An instance definition for type constructor t may formulate class constraints for its argument types a, b, \dots:\\
\codeline{instance (C\_1 a, C\_2 a, \dots) => C t where}


% \codefile{haskell}{caption={rock-paper-scissor-inst.hs}, label=rock-paper-scissor-inst}{../material/rock-paper-scissor-inst.hs}


\subsection{Deriving Class Instances}

Automatically make user-defined data types (\codeline{data \dots}) instances of classes C\_i $\in$ \{ Eq, Ord, Enum, Bounded, Show, Read \}:

\begin{codebox}[haskell]
data T a_1 a_1 ... a_n = ...
                       | ...
    deriving (C_1, C_2, ...)
\end{codebox}

% \codefile{haskell}{caption={rock-paper-scissor-inst-deriving.hs}, label=rock-paper-scissor-inst-deriving}{../material/rock-paper-scissor-inst-deriving.hs}




% chapter type_classes (end)

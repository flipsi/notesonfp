% Author: Philipp Moers <soziflip funny character gmail dot com>

\chapter{Values and Types} % (fold)
\label{cha:values_and_types}

Any Haskell expression e has a type t (\codeline{e :: t}) that is determined at compile time.
The \textbf{type assigmnent ::} is either given explicitly or inferred by the compiler.


\section{Base Types}

\vspace{9pt}\begin{center}\begin{tabular}{|c|c|c|}\hline
\rowcolor{grau}     Type            & Description                                   & values                                \\\hline
                    Int             & fixed-prec. integer                           & 0, 1, (-42)                           \\\hline
                    Integer         & arbitrary prec. integer                       & 10\textasciicircum 100                \\\hline
                    Float, Double   & single/double floating point (IEEE)           & 0.1, 1e02                             \\\hline
                    Char            & Unicode character                             & 'x', '\textbackslash t', 
                                                                                        '$\triangle$', '\textbackslash 8710'\\\hline
                    Bool            & Boolean                                       & True, False                           \\\hline
                    ()              & Unit                                          & ()                                    \\\hline
\end{tabular}\end{center}\vspace{9pt}


\section{Type Constructors}

\begin{itemize}
    \item Build new types from existing types
    \item Let a, b \dots denote arbitrary types (\textbf{type variables})
\end{itemize}

\vspace{9pt}\begin{center}\begin{tabular}{|c|c|c|}\hline
\rowcolor{grau}     Type            & Description                                   & values                        \\\hline
                    (a, b)          & pairs of values of type a, b                  & (1, True) :: (Int, Bool)      \\\hline
                    (a$_1$, a$_2$, \dots a$_n$) & n-tuples                          &                               \\\hline
                    [a]             & list of values of type a                      & [True, False] :: [Bool], []::[a]          \\\hline
                    Maybe a         & optional value of type a                      & \multirow{2}{3.7cm}{Just 42 :: Maybe Int 
                                                                                                         Nothing :: Maybe a}    \\
                                    &                                               &                               \\\hline
                    Either a b      & choice                                        & \multirow{2}{5cm}{Left 'x' :: Either Char b
                                                                                                     Right pi :: Either a Double}   \\
                                    &                                               &                               \\\hline
                    IO a            & \multirow{2}{4.2cm}{I/O actions that 
                                                            return a value of type a}   & print 42 :: IO ()             \\
                                    &                                               &                               \\\hline
                    a ->\ b& functions from a to b                       & isLetter :: Char ->\ Bool      \\\hline
\end{tabular}\end{center}\vspace{9pt}


\section{Currying}

\begin{itemize}
  \item Recall: \codeline{e$_1$ ++ e$_2$} $\equiv$ \codeline{(++) e$_1$ e$_2$}
  \item \codeline{(++) e$_1$ e$_2$} $\equiv$ \codeline{((++) e$_1$) e$_2$}
  \item Function application happens one argument at a time. \\ (\textbf{Currying}, Haskell B. Curry)
  \item Type of n-ary function is \\ a$_1$ -> a$_2$ -> \dots a$_n$ -> b
  \item Type fun -> associates to the right, read above type as \\ a$_1$ -> (a$_2$ -> (\dots ($a_n$ -> $b$)))
  \item Enables \textbf{Partial Application}
\end{itemize}


\section{Defining Values (and thus functions)}

\begin{itemize}
  \item \codeline{=} binds names to values. Names must not start with A-Z (Haskell style: camelCase)
  \item Define constant (0-ary function) c. Value of c is value of expression e. \\ \codeline{c = e}
  \item Define n-ary function f with arguments x$_i$. f may occur in e. \\ \codeline{f x$_1$ x$_2$ \dots x$_n$ = e}
  \item A Haskell program is a set of bindings.
  \item Good style: give type assigmnents for top-level (global) bindings: 
  \begin{codebox}[haskell]
f :: a_1 -> a_2 -> b
f x_1 x_2 = e
  \end{codebox}
\end{itemize}

\subsection{Guards}

Guards are conditional expressions (something like 'switch' in Java).
They are a lot more readable and more powerful than \codeline{if \dots then \dots else \dots}.

Guards are introduced by \codeline{|}:
\begin{codebox}[haskell]
f x_1 x_2 ... x_n
  | q_1     = e_1
  | q_2     = e_2
  ...
  | q_m     = e_m
[ | otherwise   = e_m+1 ]
\end{codebox}

Guards (q\_i) are expressions of type Bool, evaluated top to bottom.

\codefile{haskell}{caption={factorial.hs}, label=factorial}{../material/factorial.hs}



\subsection{Local Definitions}

\begin{enumerate}
  \item \textbf{Where bindings}: local definitions visible in the entire rhs of a definition.\\
  \begin{codebox}[haskell]
f_1 x_1 x_2 ... x_n | q_1 = e_1
                    | q_2 = e_2 
                    ...
                    | q_m = e_m 
          where 
              g_1 = ...
              g_2 = ...
              ...
              g_o
  \end{codebox}

  \codefile{haskell}{caption={power.hs}, label=power}{../material/power.hs}

  \item \textbf{Let expressions}: local definitions visible inside one expression.\\
  \begin{codebox}[haskell]
let g_1 = ...
    g_2 = ...
    ...
    g_o
in e
  \end{codebox}
\end{enumerate}


\subsection{Lists}

\begin{itemize}
  \item Recursive definitions:
  \begin{enumerate}
      \item \codeline{[]} is a list (nil), type [] :: [a]
      \item \codeline{x:xs} is a list, if x :: a, xs :: [a] \\ (x is head, xs is tail)
  \end{enumerate}
  \item Notation: \codeline{3:(2:(1:[]))} $\equiv$ \codeline{3:2:1:[]} $\equiv$ \codeline{[3,2,1]} $\equiv$ \codeline{3:[2,1]}
  \item Law: $\forall$ xs :: [a] :   \hspace{1cm} (xs $\neq$ []) \\
      \codeline{head xs : tail xs} == xs
\end{itemize}




\subsection{Pattern Matching}

\begin{itemize}
  \item \textit{The} idiomatic Haskell way to define a function by cases:
  \begin{codebox}[haskell]
f :: a_1 -> ... a_n -> b
f p_11 ... p_1k = e_1
f p_21 ... p_2k = e_2
...
f p_n1 ... p_nk = e_k
  \end{codebox}

\end{itemize}

\vspace{9pt}\begin{center}\begin{tabular}{|c|c|c|}\hline
\rowcolor{grau}   Pattern         & Matches If                & Bindings in e$_r$     \\\hline
                  constant c      & x$_i$ == c                  &                     \\\hline
                  variable v      & always                    & v $\equiv$ x$_i$      \\\hline
                  wildcard \_      & always                    &                       \\\hline
                  tuple (p$_1$, \dots p$_m$)  & components of x$_i$ match patterns p    & \\\hline
                  []              & x$_i$ == []                 &                     \\\hline
                  (p$_1$ : p$_2$)     & head x$_i$ matches p$_1$, tail x$_i$ matches p$_2$    & \\\hline
\end{tabular}\end{center}\vspace{9pt}


\codefile{haskell}{caption={tally.hs}, label=tally}{../material/tally.hs}
\codefile{haskell}{caption={take.hs}, label=take}{../material/take.hs}
\codefile{haskell}{caption={mergesort.hs}, label=mergesort}{../material/mergesort.hs}




% chapter values_and_types (end)





%!TEX root = fp.tex

% Author: Philipp Moers <soziflip funny character gmail dot com>

\chapter{Monads} % (fold)
\label{cha:monads}


\begin{codebox}[haskell]
(.) :: (b -> c) -> (a -> b) -> (a -> c)
f . g = \x -> f (g x)
(>=>) :: (a -> b) -> (b -> c) -> (a -> c)
f >=> g = g . f
\end{codebox}
\begin{itemize}
    \item \codeline{f$_1$ >=> f$_2$ >=> \dots >=> f$_n$} composes the function in left-to-right order (think UNIX pipes)
    \item \codeline{(>=>, id)} forms a monoid: \codeline{>=>} is associative with identify id.
\end{itemize}


\section{Sequencing of partial functions} 
(\codeline{a -> Maybe b})

\begin{itemize}
    \item Return Nothing as soon as first function in sequence yields Nothing:\\
        \begin{codebox}[haskell]
(>=>) :: (a -> Maybe b) -> (b -> Maybe c) -> (a -> Maybe c)
f >=> g = \x -> case f x of
        Nothing -> Nothing
        Just y  -> g y
        \end{codebox}
\end{itemize}

\codefile{haskell}{caption={sequence-Maybe.hs}, label={sequence-Maybe.hs}}{../material/sequence-Maybe.hs_ASCII}


\section*{Sequencing exception-generating functions \codeline{a -> Exc b}}

\codeline{type Exc b = Either Error b}
 
\begin{itemize}
    \item Define suitable representation type for Error (\codeline{type Error = String})
    \item Exceptions are propagated once occured:
\begin{codebox}[haskell]
f >=> g = \x -> case f x of
    Left err -> Left err
    Right y  -> g y
\end{codebox}
\end{itemize}

\codefile{haskell}{caption={sequence-Either.hs}, label={sequence-Either.hs}}{../material/sequence-Either.hs_ASCII}


\section*{Sequencing non-deterministic functions \codeline{a -> NonDet b}}

\begin{itemize}
    \item Non-determinism any answer in a \underline{list} of answers:\\
    \codeline{type NonDet b = [b]}
    \item Take all possible answers into account:\\
\begin{codebox}[haskell]
(>=>) :: (a -> NonDet b) -> (b -> NonDet c) -> (a -> NonDet c)
f >=> g = \x -> concat [ g y | y <- f x ]
\end{codebox}
\end{itemize}


\section*{Sequencing stateful functions \codeline{a -> State -> (b, State)}}

\begin{itemize}
    \item State Transformer: \codeline{type ST b = State -> (b, State)}
    \item Stateful function: \codeline{a -> ST b}
    \item Define suitable representation of state State
\begin{codebox}[haskell]
(>=>) :: (a -> ST b) -> (b -> ST c) -> (a -> ST c)
f >=> g = \x s0 -> let (y, s1) = f x s0
                   in g y s1
\end{codebox}
\end{itemize}



\codefile{haskell}{caption={sequence-ST.hs}, label={sequence-ST.hs}}{../material/sequence-ST.hs_ASCII}


\section*{Sequencing side-effecting functions \codeline{a -> World -> (b, World)}}

\begin{itemize}
    \item Side effects: functions consume current ``world state'', return new state along with result.
\begin{codebox}[haskell]
type World = ... -- abstract (defined by Haskell's runtime)
type IO b = World -> (b, World)
\end{codebox}
    \item Value of type IO b is an action that
    \begin{enumerate}
        \item \underline{when performed} has a side effect on the world
        \item returns a value of type b
    \end{enumerate}
    \item Haskell built-ins:
    \begin{itemize}
        \item \codeline{print :: Show a => a -> IO ()}
        \item \codeline{putStrLn :: String -> IO ()}
        \item \codeline{getLine :: IO String}
        \item \codeline{readFile :: FilePath -> IO String}
        \item \codeline{(>=>) :: (a -> IO b) -> (b -> IO c) -> (a -> IO c)}
    \end{itemize}
\end{itemize}


\section*{Abstraction of Sequencing functions}

\codeline{(>=>) :: (a -> m b) -> (b -> m c) -> (a -> m c)}

\begin{itemize}
    \item Type class \textbf{Monad}
\begin{codebox}[haskell]
class Monad m where
    return :: a -> m 
    (>>=) :: m a -> (a -> m b) -> m b
\end{codebox}
\end{itemize}

\vspace{9pt}\begin{center}\begin{tabular}{|c|c|}\hline
\rowcolor{grau} Monad m             & return x (lifting)    \\\hline
                Maybe               & Just x                \\\hline
                Exc                 & Right x               \\\hline
                NonDet              & [x]                   \\\hline
                ST                  & \textbackslash s -> (x, s)          \\\hline
\end{tabular}\end{center}\vspace{9pt}

\begin{codebox}[haskell]
instance Monad Maybe where
    return x = Just x
    Nothing >>= g = Nothing
    Just y  >>= g = g y
\end{codebox}

\subsection*{Kleisli Composition}
\begin{codebox}[haskell]
(>=>) :: Monad m => (a -> m b) -> (b -> m c) -> (a -> m c)
f >=> g = \x -> f x >>= g
\end{codebox}

\codefile{haskell}{caption={monadic-Maybe.hs}, label={monadic-Maybe.hs}}{../material/monadic-Maybe.hs_ASCII}
\codefile{haskell}{caption={monadic-Either.hs}, label={monadic-Either.hs}}{../material/monadic-Either.hs_ASCII}
\codefile{haskell}{caption={monadic-ST.hs}, label={monadic-ST.hs}}{../material/monadic-ST.hs_ASCII}
\codefile{haskell}{caption={monadic-NonDet.hs}, label={monadic-NonDet.hs}}{../material/monadic-NonDet.hs_ASCII}

\newpage

\section*{do-Notation}

\begin{itemize}
    \item (\codeline{e}: expr. of type \codeline{m a}, \codeline{es}: sequence of \codeline{;}-separated expressions)
    \begin{codebox}[haskell]
    do {e} @$\equiv$@ e
    do {x <- e; es} @$\equiv$@
        e >>= \x -> do {es}
    do {e; es} @$\equiv$@
        e >>= \_ -> do {es} -- >>
    do {let v=x; es} @$\equiv$@ let v = x in do {es}
    \end{codebox}

    \item \codeline{do \{\dots\} :: m b}
    \item May use layout instead of \codeline{\{\dots\}} and \codeline{;}.

\end{itemize}

\codefile{haskell}{caption={monadic-do-NonDet.hs}, label={monadic-do-NonDet.hs}}{../material/monadic-do-NonDet.hs}


\section*{Input/Output}
Key idea: Decouple the \underline{pure} construction of I/O actions from their \underline{effectful} execution.
\begin{itemize}
    \item Construct (complex) I/O actions of type \codeline{IO a} using \codeline{getLine}, \codeline{putStrLn}, \codeline{>>=}. \underline{No I/O or side effects happened yet.}
    \item Execute constructed IO actions. Side effects will occur and a value of type a will be returned. \underline{Only} the IO action returned by function \codeline{main} is executed by Haskell's runtime system.
\end{itemize}

\codefile{haskell}{caption={IOaction.hs}, label={IOaction.hs}}{../material/IOaction.hs_ASCII}

\textit{\href{http://www.google.com/search?q=free+monads}{google ``free monads''}}


\section*{Monad Laws}

\begin{enumerate}
    \item \codeline{return x >>= g} $\equiv$ \codeline{g x}
    \item \codeline{m >>= return} $\equiv$ \codeline{m}
    \item \codeline{m >>= f >>= g} $\equiv$ m >>= (\textbackslash x -> f x >>= g)
\end{enumerate}

% \codefile{haskell}{caption={monad-laws.lhs}, label={monad-laws.lhs}}{../material/monad-laws.lhs}


\section*{Another Monad}

\begin{codebox}[haskell]
data Perhaps a = Perhaps { value :: a, prohability :: Prob}
deriving (Show)
\end{codebox}

Generalized \codeline{Maybe a}
\begin{itemize}
    \item \codeline{Just x} $\equiv$ \codeline{always x = Perhaps x 1.0}
    \item \codeline{Nothing} $\equiv$ \codeline{never = Perhaps undefined 0.0}
\end{itemize}
\codeline{newtype Prob = P Float}

\begin{itemize}
    \item Language extension \emph{GeneralizedNewtypeDeriving}: newtype of the form \\ \codeline{newtype T = k a} may inherit instances from its representations type a, eg.
    \begin{codebox}[haskell]
    newtype Prob = P Float
        deriving (Eq, Ord, Num, Real, Fractional, RealFrac)
    \end{codebox}
\end{itemize}

\codefile{haskell}{caption={Perhaps.hs}, label={Perhaps.hs}}{../material/Perhaps.hs}
\codefile{haskell}{caption={hiking.hs}, label={hiking.hs}}{../material/hiking.hs}


\subsection*{Combination of Monads}

\begin{codebox}[haskell]
a -> Maybe (Perhaps b)
a -> [Perhaps b] -- probability distribution
\end{codebox}

% \codefile{haskell}{caption={MayP-Dist}, label={MayP-Dist}}{../material/MayP-Dist}

\textbf{Monad Transformer}
\begin{itemize}
    \item Define \codeline{PerhapsT m a} which wraps (arbitrary) monad m around the inner monad Perhaps. PerhapsT is called a \textbf{monad transformer}.
\end{itemize}
\codefile{haskell}{caption={Probability.hs}, label={Probability.hs}}{../material/Probability.hs}














% chapter monads (end)





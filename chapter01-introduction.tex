% Author: Philipp Moers <soziflip funny character gmail dot com>


\chapter{Introduction} % (fold)
\label{cha:introduction}

Computational model in Functional Programming: \textbf{reduction} (replace expression to values)

In Functional Programming, expressions are formed by applying functions to values.

\begin{enumerate}
    \item Functions as in math: $x = y \Rightarrow f(x) = f(y)$
    \item Functions are values (just like numbers, text \dots)
\end{enumerate}

\vspace{9pt}\begin{center}\begin{tabular}{|c|c|c|}\hline
\rowcolor{grau}                         & Functional                                & Imperative        \\\hline
                program construction    & function application and composition      & statement sequencing      \\\hline
                execution               & reduction (expression evaluation)         & state changes             \\\hline
                semantics               & lambda calculus                           & complex (denotational)    \\\hline
\end{tabular}\end{center}\vspace{9pt}

\newpage

\subsection*{Example}
$n \in \N, n \ge 2 $ is a prime number \textit{iff} the set of non-trivial factors is empty:\\
$$ n \text{ is prime} \Leftrightarrow \{\ m\ |\ m \in \{2,\dots, n-1\},\ n \mod m = 0 \} = \emptyset $$
\\
\codefile{haskell}{caption={isPrime.hs}, label=isPrime}{../material/isPrime.hs}



% chapter introduction (end)

%!TEX root = fp.tex

% Author: Philipp Moers <soziflip funny character gmail dot com>


\chapter{Domain-Specific Languages} % (fold)
\label{cha:domain_specific_languages}
a.k.a. \textbf{DSLs}

\begin{itemize}
    \item ``small'''' languages designed to easily and directly express the concepts/idioms of a specific domain. \underline{Not} turing-complete in general.
    \item Examples:
        \vspace{9pt}\begin{center}\begin{tabular}{|c|c|}\hline
        \rowcolor{grau} Domain          & DSLs                                  \\\hline
                        OS automation   & shell scripts, OSX Automater          \\\hline
                        Typesetting     & \LaTeX                                \\\hline
                        Queries         & SQL                                   \\\hline
                        Game Scripting  & Unreal Script, Lua                    \\\hline
                        Parsing         & Yacc, Bison, ANTLR                    \\\hline
        \end{tabular}\end{center}\vspace{9pt}
    \item Functional Languages make good hosts for \textbf{embedded DSLs}:
        \begin{itemize}
            \item algebraic datatypes (e.g. to model ASTs)
            \item higher-order functions (abstractionn, control constructs)
            \item lightweight syntax (layout / whitespace, non-alphabetic ids)
        \end{itemize}
\end{itemize}


\newpage

\textit{Example}: An embedded DSL for integer sets:
\begin{codebox}[haskell]
type IntegerSet = 
    -- constructors:
    empty :: IntegerSet
    insert :: Integer -> IntegerSet -> IntegerSet
    delete :: Integer -> IntegerSet -> IntegerSet
    -- observer:
    member :: Integer -> IntegerSet -> Bool

member 3 (insert 1 (delete 3 (insert 2 (insert 3 empty))))
    @$\equiv$@ False
\end{codebox}

\vspace{2cm}

\textbf{(1)}
DSL as library of functions, implementation details exposed. 

\vspace{9pt}
% TODO encode unicode chars!!!
% \codefile{haskell}{caption={library-exposed.hs}, label=library-exposed.hs}{../material/library-exposed.hs}
\textcolor{myorange}{[Here is library-exposed.hs missing!]}
\vspace{9pt}



\section{Modules}

\begin{itemize}
    \item Group of related definitions (values, types) in a single file \\ (named ``M.hs'' / ``M.lhs''):
    \begin{codebox}[haskell]
module M where
    type Predicate a = a -> Bool
    id :: a -> a
    id x = x
    \end{codebox}
    \item Hierarchy: module A.B.C.M in file A/B/C/M.hs
    \item Access definitions in other module M:
    \begin{codebox}[haskell]
import M
    \end{codebox}

    \newpage
    \item Explicit export lists hide all other definitions:
    \begin{codebox}[haskell]
module M (id) where
    ...
    -- type Predicate a not exported
    \end{codebox}
    \item \textbf{Abstract data types}:\\
    export algebraic data types, but \underline{not} its constructors:
        \begin{codebox}[haskell]
module M (Rose, leaf) where
    data Rose a = Node a [Rose a]
    leaf :: a -> Rose a [Rose a]
    leaf x = Node x []
        \end{codebox}
    \begin{itemize}
        \item Export constructors:
        \begin{codebox}[haskell]
module M (Rose(Node), leaf) where
    @\dots@
-- or export all constructors:
module M (Rose(..), leaf) where
        \end{codebox}
        \item Instance definitions (including deriving) are exported with their type.
    \end{itemize}
    \item Qualified imports to partition name space:
    \begin{codebox}[haskell]
import qualified M
    ...
    -- use M.foobar syntax
    t :: M.Rose Char
    t = M.leaf 'x'
    \end{codebox}
    \item Partially import module:
    \begin{codebox}[haskell]
-- only import nub and reverse
import Data.List (nub, reverse)

-- import whole module but without otherwise
import Prelude hiding (otherwise)
otherwise :: Bool
otherwise = False -- harhar
    \end{codebox}
    \item Pass imported modules to importer of own module:
    \begin{codebox}[haskell]
module M (..., module Data.List, ...) where
    import Data.List (nub)
    \end{codebox}
    \begin{codebox}[haskell]
import qualified M
    M.nub
    \end{codebox}
\end{itemize}

% \codefile{haskell}{caption={SetLanguage.hs}, label=SetLanguage.hs}{../material/SetLanguage.hs}
% \codefile{haskell}{caption={SetLanguageShallow.hs}, label=SetLanguageShallow.hs}{../material/SetLanguageShallow.hs}
\codefile{haskell}{caption={SetLanguageShallowCard.hs}, label=SetLanguageShallowCard.hs}{../material/SetLanguageShallowCard.hs}
\codefile{haskell}{caption={set-language-shallow.hs}, label=set-language-shallow.hs}{../material/set-language-shallow.hs}


\vspace{2cm}

\textbf{(2)}
DSL as library of functions, abstract data type (module).
\begin{itemize}
    \item \textbf{Shallow DSL embedding}:\\
    Semantics of DSL operations directly expressed in terms of host language value (e.g. list or characteristic function)
    \begin{itemize}
        \item constructors (empty, insert, delete) perform the work, harder to add
        \item observers (member) trivial
    \end{itemize}
    \item \textbf{Deep DSL embedding}:\\
    DSL operations build an abstract syntax tree (AST) that represents applications and arguments
    \begin{itemize}
        \item constructors merely build the AST, very easy to add
        \item observers interpret (traverse) the AST and perform the work
    \end{itemize}
    % \codefile{haskell}{caption={SetLanguageDeep.hs}, label=SetLanguageDeep.hs}{../material/SetLanguageDeep.hs}
    \codefile{haskell}{caption={SetLanguageDeepCard.hs}, label=SetLanguageDeepCard.hs}{../material/SetLanguageDeepCard.hs}
    \codefile{haskell}{caption={set-language-deep.hs}, label=set-language-deep.hs}{../material/set-language-deep.hs}

\end{itemize}

\newpage 

\codefile{haskell}{caption={ExprDeepNum.hs}, label={ExprDeepNum.hs}}{../material/ExprDeepNum.hs}
\codefile{haskell}{caption={expr-deep-num.hs}, label={expr-deep-num.hs}}{../material/expr-deep-num.hs}

% \codefile{haskell}{caption={expr-language.hs}, label={expr-language.hs}}{../material/expr-language.hs}


% \codefile{haskell}{caption={ExprDeep.hs}, label={ExprDeep.hs}}{../material/ExprDeep.hs}
% \codefile{haskell}{caption={ExprDeepGADTUntyped.hs}, label={ExprDeepGADTUntyped.hs}}{../material/ExprDeepGADTUntyped.hs}



\section{Generalized Algebraic Data Types (GADTs)}

\textit{Idea}: 
\begin{itemize}
    \item Encode the type of a DSL expression (here: Integer or Bool) in its \textbf{Haskell representation type}.
    \item Use Haskell's type checker to ensure at compile time that only well-typed DSL expressions are built.
\end{itemize}

\textit{Language extensions:}\\
\codeline{\{-\# LANGUAGE GADTs \#-\}}
\begin{itemize}
    \item Define new parameterized type T, its constructors k$_i$ and their type signatures:
    \begin{codebox}[haskell]
data T a@$_1$@ a@$_2$@ @\dots@ a@$_n$@ where
    k@$_1$@ :: b@$_{11}$@ -> @\dots@ -> b@$_1n_1$@ -> T t@$_{11}$@ t@$_{1n}$@
    @\dots@
    k@$_r$@ :: b@$_{r1}$@ -> @\dots@ -> b@$_rn_r$@ -> T t@$_{r1}$@ t@$_{rn}$@
    \end{codebox}
\end{itemize}

\codefile{haskell}{caption={ExprDeepGADTTyped.hs}, label={ExprDeepGADTTyped.hs}}{../material/ExprDeepGADTTyped.hs}
\codefile{haskell}{caption={expr-language-typed.hs}, label={expr-language-typed.hs}}{../material/expr-language-typed.hs}

% \codefile{haskell}{caption={ExprShallow.hs}, label={ExprShallow.hs}}{../material/ExprShallow.hs}
% \codefile{haskell}{caption={expr-language-shallow.hs}, label={expr-language-shallow.hs}}{../material/expr-language-shallow.hs}
\codefile{haskell}{caption={ExprShallowPrint.hs}, label={ExprShallowPrint.hs}}{../material/ExprShallowPrint.hs}
\codefile{haskell}{caption={expr-language-shallow-print.hs}, label={expr-language-shallow-print.hs}}{../material/expr-language-shallow-print.hs}



\section{Shallow embedding of a String Matching DSL}

\begin{itemize}
    \item \textbf{Pattern}:
    \begin{enumerate}
        \item Given a String, a pattern returns the list of matches. Match failure? Returns the empty list.
        \item A match consists of a value (e.g. the matched characters, tokens, parse trees) and the residual String left to match.
    \end{enumerate}
    Thus: \codeline{type Pattern a = String -> [(a, String)]}
\end{itemize}

\subsection{DSL design}
\vspace{9pt}\begin{center}\begin{tabular}{|c|c|c|}\hline
\rowcolor{grau} Pattern                 &               & DSL function     \\\hline
                match lit. char         & "x"           & \codeline{lit :: Char -> Pattern Char}    \\\hline
                match empty string      & $\epsilon$    & \codeline{empty :: a -> Pattern a}        \\\hline
                fail always             & $\emptyset$   & \codeline{fail :: Pattern a}              \\\hline
                alternative             & |             & \parbox[t]{9cm}{\codeline{alt :: Pattern a ->} \\ \codeline{Pattern a -> Pattern a}}  \\\hline
                sequence                & .             & \parbox[t]{9cm}{\codeline{seq :: (a -> b -> c) ->} \\ \codeline{Pattern a -> Pattern b -> Pattern c}} \\\hline
                repitition              & *             & \codeline{rep :: Pattern a -> Pattern [a]} \\\hline
\end{tabular}\end{center}\vspace{9pt}

% \codefile{haskell}{caption={PatternMatch.hs}, label={PatternMatch.hs}}{./PatternMatch.hs}
% \codefile{haskell}{caption={pattern-match.hs}, label={pattern-match.hs}}{./pattern-match.hs}
\codefile{haskell}{caption={PatternMatching.hs}, label={PatternMatching.hs}}{../material/PatternMatching.hs}
\codefile{haskell}{caption={simple-patterns.hs}, label={simple-patterns.hs}}{../material/simple-patterns.hs}

\codefile{haskell}{caption={pattern-matching.hs}, label={pattern-matching.hs}}{../material/pattern-matching.hs}












% chapter domain_specific_languages (end)

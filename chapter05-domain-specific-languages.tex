%!TEX root = fp.tex

% Author: Philipp Moers <soziflip funny character gmail dot com>


\chapter{Domain-Specific Languages} % (fold)
\label{cha:domain_specific_languages}
a.k.a. \textbf{DSLs}

\begin{itemize}
    \item ``small'''' languages designed to easily and directly express the concepts/idioms of a specific domain. \underline{Not} turing-complete in general.
    \item Examples:
        \vspace{9pt}\begin{center}\begin{tabular}{|c|c|}\hline
        \rowcolor{grau} Domain          & DSLs                                  \\\hline
                        OS automation   & shell scripts, OSX Automater          \\\hline
                        Typesetting     & \LaTeX                                \\\hline
                        Queries         & SQL                                   \\\hline
                        Game Scripting  & Unreal Script, Lua                    \\\hline
                        Parsing         & Yacc, Bison, ANTLR                    \\\hline
        \end{tabular}\end{center}\vspace{9pt}
    \item Functional Languages make good hosts for \textbf{embedded DSLs}:
        \begin{itemize}
            \item algebraic datatypes (e.g. to model ASTs)
            \item higher-order functions (abstractionn, control constructs)
            \item lightweight syntax (layout / whitespace, non-alphabetic ids)
        \end{itemize}
\end{itemize}


\newpage

\textit{Example}: An embedded DSL for integer sets:
\begin{codebox}[haskell]
type IntegerSet = 
    -- constructors:
    empty :: IntegerSet
    insert :: Integer -> IntegerSet -> IntegerSet
    delete :: Integer -> IntegerSet -> IntegerSet
    -- observer:
    member :: Integer -> IntegerSet -> Bool

member 3 (insert 1 (delete 3 (insert 2 (insert 3 empty))))
    @$\equiv$@ False
\end{codebox}

\vspace{2cm}

\textbf{(1)}
DSL as library of functions, implementation details exposed. 

\vspace{9pt}
% TODO encode unicode chars!!!
% \codefile{haskell}{caption={library-exposed.hs}, label=library-exposed.hs}{../material/library-exposed.hs}
\textcolor{myorange}{[Here is library-exposed.hs missing!]}
\vspace{9pt}



\section{Modules}

\begin{itemize}
    \item Group of related definitions (values, types) in a single file \\ (named ``M.hs'' / ``M.lhs''):
    \begin{codebox}[haskell]
module M where
    type Predicate a = a -> Bool
    id :: a -> a
    id x = x
    \end{codebox}
    \item Hierarchy: module A.B.C.M in file A/B/C/M.hs
    \item Access definitions in other module M:
    \begin{codebox}[haskell]
import M
    \end{codebox}

    \newpage
    \item Explicit export lists hide all other definitions:
    \begin{codebox}[haskell]
module M (id) where
    ...
    -- type Predicate a not exported
    \end{codebox}
    \item \textbf{Abstract data types}:\\
    export algebraic data types, but \underline{not} its constructors:
        \begin{codebox}[haskell]
module M (Rose, leaf) where
    data Rose a = Node a [Rose a]
    leaf :: a -> Rose a [Rose a]
    leaf x = Node x []
        \end{codebox}
    \begin{itemize}
        \item Export constructors:
        \begin{codebox}[haskell]
module M (Rose(Node), leaf) where
    @\dots@
-- or export all constructors:
module M (Rose(..), leaf) where
        \end{codebox}
        \item Instance definitions (including deriving) are exported with their type.
    \end{itemize}
    \item Qualified imports to partition name space:
    \begin{codebox}[haskell]
import qualified M
    ...
    -- use M.foobar syntax
    t :: M.Rose Char
    t = M.leaf ``x''
    \end{codebox}
    \item Partially import module:
    \begin{codebox}[haskell]
-- only import nub and reverse
import Data.List (nub, reverse)

-- import whole module but without otherwise
import Prelude hiding (otherwise)
otherwise :: Bool
otherwise = False -- harhar
    \end{codebox}
    \item Pass imported modules to importer of own module:
    \begin{codebox}[haskell]
module M (..., module Data.List, ...) where
    import Data.List (nub)
    \end{codebox}
    \begin{codebox}[haskell]
import qualified M
    M.nub
    \end{codebox}
\end{itemize}

% \codefile{haskell}{caption={SetLanguage.hs}, label=SetLanguage.hs}{../material/SetLanguage.hs}
% \codefile{haskell}{caption={SetLanguageShallow.hs}, label=SetLanguageShallow.hs}{../material/SetLanguageShallow.hs}
\codefile{haskell}{caption={SetLanguageShallowCard.hs}, label=SetLanguageShallowCard.hs}{../material/SetLanguageShallowCard.hs}
\codefile{haskell}{caption={set-language-shallow.hs}, label=set-language-shallow.hs}{../material/set-language-shallow.hs}


\vspace{2cm}

\textbf{(2)}
DSL as library of functions, abstract data type (module).
\begin{itemize}
    \item \textbf{Shallow DSL embedding}:\\
    semantics of DSL operations directly expressed in terms of host language value (e.g. list or characteristic function)
    \begin{itemize}
        \item constructors (empty, insert, delete) perform the work, harder to add
        \item observers (member) trivial
    \end{itemize}
    \item \textbf{Deep DSL embedding}:\\
    DSL operations build an abstract syntax tree (AST) that represents applications and arguments
    \begin{itemize}
        \item constructors merely build the AST, very easy to add
        \item observers interpret (traverse) the AST and perform the work
    \end{itemize}
    % \codefile{haskell}{caption={SetLanguageDeep.hs}, label=SetLanguageDeep.hs}{../material/SetLanguageDeep.hs}
    \codefile{haskell}{caption={SetLanguageDeepCard.hs}, label=SetLanguageDeepCard.hs}{../material/SetLanguageDeepCard.hs}
    \codefile{haskell}{caption={set-language-deep.hs}, label=set-language-deep.hs}{../material/set-language-deep.hs}

\end{itemize}




% chapter domain_specific_languages (end)
